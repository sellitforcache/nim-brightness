%% This is file `elsarticle-template-1-num.tex',
%%
%% Copyright 2009 Elsevier Ltd
%%
%% This file is part of the 'Elsarticle Bundle'.
%% ---------------------------------------------
%%
%% It may be distributed under the conditions of the LaTeX Project Public
%% License, either version 1.2 of this license or (at your option) any
%% later version.  The latest version of this license is in
%%    http://www.latex-project.org/lppl.txt
%% and version 1.2 or later is part of all distributions of LaTeX
%% version 1999/12/01 or later.
%%
%% The list of all files belonging to the 'Elsarticle Bundle' is
%% given in the file `manifest.txt'.
%%
%% Template article for Elsevier's document class `elsarticle'
%% with numbered style bibliographic references
%%
%% $Id: elsarticle-template-1-num.tex 149 2009-10-08 05:01:15Z rishi $
%% $URL: http://lenova.river-valley.com/svn/elsbst/trunk/elsarticle-template-1-num.tex $
%%
\documentclass[preprint,12pt]{elsarticle}
\usepackage{amsmath}
\usepackage{graphicx,subcaption}
\usepackage{listings}
\usepackage{hyperref}
\usepackage{multirow}
\usepackage{notoccite}
\hypersetup{
    colorlinks,
    citecolor=black,
    filecolor=black,
    linkcolor=black,
    urlcolor=black
}

%% Use the option review to obtain double line spacing
%% \documentclass[preprint,review,12pt]{elsarticle}

%% Use the options 1p,twocolumn; 3p; 3p,twocolumn; 5p; or 5p,twocolumn
%% for a journal layout:
%% \documentclass[final,1p,times]{elsarticle}
%% \documentclass[final,1p,times,twocolumn]{elsarticle}
%% \documentclass[final,3p,times]{elsarticle}
%% \documentclass[final,3p,times,twocolumn]{elsarticle}
%% \documentclass[final,5p,times]{elsarticle}
%% \documentclass[final,5p,times,twocolumn]{elsarticle}

%% if you use PostScript figures in your article
%% use the graphics package for simple commands
%% \usepackage{graphics}
%% or use the graphicx package for more complicated commands
%% \usepackage{graphicx}
%% or use the epsfig package if you prefer to use the old commands
%% \usepackage{epsfig}

%% The amssymb package provides various useful mathematical symbols
\usepackage{amssymb}
%% The amsthm package provides extended theorem environments
%% \usepackage{amsthm}

%% The lineno packages adds line numbers. Start line numbering with
%% \begin{linenumbers}, end it with \end{linenumbers}. Or switch it on
%% for the whole article with \linenumbers after \end{frontmatter}.
\usepackage{lineno}

%% natbib.sty is loaded by default. However, natbib options can be
%% provided with \biboptions{...} command. Following options are
%% valid:

%%   round  -  round parentheses are used (default)
%%   square -  square brackets are used   [option]
%%   curly  -  curly braces are used      {option}
%%   angle  -  angle brackets are used    <option>
%%   semicolon  -  multiple citations separated by semi-colon
%%   colon  - same as semicolon, an earlier confusion
%%   comma  -  separated by comma
%%   numbers-  selects numerical citations
%%   super  -  numerical citations as superscripts
%%   sort   -  sorts multiple citations according to order in ref. list
%%   sort&compress   -  like sort, but also compresses numerical citations
%%   compress - compresses without sorting
%%
%% \biboptions{comma,round}

% \biboptions{}

\let\originaleqref\eqref
\renewcommand{\eqref}{Eq.~\originaleqref}

\hypersetup{colorlinks=true,
  pdftitle={Brightness of the liquid determium cold source measured from the ICON beamline at the Swiss Spallation Neutron Source (SINQ)},
  pdfauthor={Ryan M. Bergmann, Masako Yamada, Tibor Reiss, Michael Wohlmuther, Uwe Filges}}

\journal{Nuclear Instruments and Methods in Physics Research Section A: Accelerators, Spectrometers, Detectors and Associated Equipment}

\begin{document}

\begin{frontmatter}

%% Title, authors and addresses

%% use the tnoteref command within \title for footnotes;
%% use the tnotetext command for the associated footnote;
%% use the fnref command within \author or \address for footnotes;
%% use the fntext command for the associated footnote;
%% use the corref command within \author for corresponding author footnotes;
%% use the cortext command for the associated footnote;
%% use the ead command for the email address,
%% and the form \ead[url] for the home page:
%%
%% \title{Title\tnoteref{label1}}
%% \tnotetext[label1]{}
%% \author{Name\corref{cor1}\fnref{label2}}
%% \ead{email address}
%% \ead[url]{home page}
%% \fntext[label2]{}
%% \cortext[cor1]{}
%% \address{Address\fnref{label3}}
%% \fntext[label3]{}

\title{Brightness of the liquid determium cold source measured from the ICON beamline at the Swiss Spallation Neutron Source (SINQ)}

%% use optional labels to link authors explicitly to addresses:
%% \author[label1,label2]{<author name>}
%% \address[label1]{<address>}
%% \address[label2]{<address>}

\author{Ryan M. Bergmann\corref{rmb}}
\ead{ryan.bergmann@psi.ch}
\cortext[rmb]{Corresponding author. Tel.: +41.56.310.56.12.}

\author{Masako Yamada}
\ead{masako.yamada@psi.ch}



\address{Paul Scherrer Institut, Villigen, Switzerland}

\begin{abstract}


\end{abstract}

\begin{keyword}
brightness \sep cold


\end{keyword}


\end{frontmatter}

\linenumbers

%% main text

\section{Introduction}
\label{sec:intro}

The Swiss Spallation Neutron Source (SINQ) is a spallation neutron source driven by a continuous 590 MeV proton beam at the Paul Scherrer Institut in Villigen, Switzerland.  The incoming protons impinge on a lead target cooled with heavy water, producing high energy neutrons.  These neutrons are moderated by the tank of D$_2$O surrounding the target.  The cold source is a 20 liter volume of liquid D$_2$ at approximately 25 K located within the D$_2$O moderator tank with its innermost face approximately 35 cm away from the center of the target.  It serves to further reduce the energies of the ``thermal'' neutrons coming from the D$_2$O tank into a regime that is more useful to instruments which receive neutrons from its surface.  

There is an extended shutdown of SINQ ambitiously planned for 2018 which will provide an opportunity for changes to be made to the cold neutron source \cite{rueegg_icans}.  Any changes will require a large amount of calculations done to minimize risk and ensure good the proposed changes will benifit the facility and its users.  The brightness measureemtn presented in this paper is the first step in validating the computational models reuqired to carry out a sucessful upgrade.  

The brightness measurement was conducted on July XXX 2014 in the ICON beamline at SINQ.  ICON is the cold neutron imaging facility at the neutron spallation source SINQ. The beamline offers an aperture wheel to change beam intensity and collimation ratio, as well as large evaculated flight tubes to minimize losses from scattering in the air \cite{icon}.  ICON was chosen since it does not incorporate any neutron optics, has good collimation, and looks directly on the surface of the cold source.  This removes additional uncertainty associated with the current state of the neutron guides and the physics involved with neutron-refecltive surfaces.

\section{Experimental Setup}
\label{sec:setup}



\section{MCNP Simulation}
\label{sec:sim}



\section{Results}
\label{sec:results}



\section{Discussion}
\label{sec:discussion}



\section{Conclusions}
\label{sec:conclusions}



\section*{Acknowledgements}
\label{sec:ack}

This work was supported by Swiss National Science Foundation grant 200021\_150048/1

\section*{Disclaimer}
\label{sec:disc}

\bibliographystyle{elsarticle-num-names.bst}
\bibliography{references}

\end{document}

